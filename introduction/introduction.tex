\chapter{Introduction générale}
Les émotions sont décrites comme des sentiments intenses dirigés à quelque chose ou quelqu'un en réponse à des événements internes ou externes ayant une signification particulière pour l'individu. Aujourd'hui, l'internet est devenu un moyen clé par lequel les gens expriment leurs émotions, leurs sentiments et leurs opinions. Chaque événement, actualité ou activité dans le monde, est partagé, discuté, publié et commenté sur les réseaux sociaux par des millions de personnes. La capture de ces émotions en texte peut être une source des informations précieuses, qui peuvent être utilisées pour étudier comment les gens réagissent à différentes situations et événements.

Les analystes commerciaux peuvent utiliser ces informations pour suivre les sentiments et les opinions des gens sur leurs produits.Les chefs d’entreprise peuvent analyser la vision globale des personnes réponse à leurs actions ou événements et s'adapter avec cette vision. Ainsi, les analystes de la santé peuvent étudier les sautes d'humeur des individus ou
masses à différents moments de la journée ou en réponse à certains événements. Ces informations peuvent également être utilisé pour formuler l'état mental d'un individu, étudiant son activité sur une période de temps, et éventuellement détecter les risques de dépression.

Le contexte de notre projet est de concevoir et réaliser une application d'analse des sentiments à partir des différentes tweets.

Ce rapport présente l'ensemble d'étapes suivies pour déveloper la solution. Il se divise en 5 chapitres organisés comme suit:
\begin{itemize}
    \item Le premier chapitre "Contexte général" discute l'état de l'art et présente le contexte général du projet.
    \item Le deuxième chapitre "Étude théorique et choix thechniques" contient une étude des solutions techniques existantes et discute les différents choix théchniques éxplorés dans notre projet.
    \item Le troisième chapitre "Analyse des besoins et spécifications" détermine les différents besoins de notre application et traite son aspect conceptuel.
    \item Le quatrième chapitre "Réalisation" illustre les différentes étapes de la réalisation de l'application.
    \item En conclusions, nous discutons les difficultés rencontrées lors de la réalisation de notre solution et les différents perspectives d'améliorations possibles.
\end{itemize}


