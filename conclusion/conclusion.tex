\newpage
%\chapter{\fontsize{14}{12}\textbf{CONCLUSION}}
\chapter*{\begin{center} Conclusion \end{center}}
    \addcontentsline{toc}{chapter}{\numberline{}Conclusion}%
Dans ce projet, nous devions concevoir et développer une application desktop d'analyse des sentiments des tweets en temps réel. \\

Dans ce rapport, nous avons commencé par une présentation du contexte générale du projet, comme on a examiné l'étude théorique effectuée lors de la réalisation en abordant l'aspect conceptuel du projet, on a aussi discuté les différents choix techniques qui nous ont aidé à l'élaboration de notre application, avant de détailler les différentes étapes de l'implémentation du projet.\\

La démarche utilisée pour résoudre ce problème a tout de même quelques limites qui
malheureusement demande plus de temps et de moyen afin de les aborder, on peut citer comme exemple le nombre des tweets fournis par TwitterAPI, qui se limite à x tweet, ainsi les humains peuvent exprimer leurs émotions de manière déguisée sans utiliser de vocabulaire émotif. Par exemple, "Mon travail m'oblige à travailler 25 heures par jour". Bien sûr, dans cette phrase, le locuteur exprime un sentiment négatif hyperbolique, suggérant qu'il doit travailler trop pour son travail. De telles expressions ne peuvent être saisies qu'à l'aide d'une base de connaissances du monde réel. En plus, la complexité des langues humains (anglaise dans notre cas) nous empêche à détecter très précisément les sentiments des tweets étudiés. \\

Dans les versions futures, Nous pourrions encore améliorer notre application en essayant d'extraire plus de fonctionnalités des tweets, en essayant différents types de fonctionnalités, en ajustant les paramètres du classificateur naïf de Bayes, ou en essayant un autre classificateur tous ensemble.



	
